\documentclass[czech,master]{resources/diploma/diploma}

% Sablonove baliky
\usepackage[autostyle=true,czech=quotes]{csquotes} % korektni sazba uvozovek, podpora pro balik biblatex
\usepackage[backend=biber, style=iso-numeric, alldates=iso]{biblatex} % bibliografie
\usepackage{dcolumn} % sloupce tabulky s ciselnymi hodnotami
\usepackage{subfig} % makra pro "podobrazky" a "podtabulky"

% Moje baliky
\usepackage{float} % lepsi umistovani obrazku (H)
\usepackage{glossaries} % balik pro praci s odbornymi pojmy
\usepackage{hyperref} % vkladani hypertextovych odkazu
\usepackage{xurl} % zalomeni dlouhych URL
\usepackage{tablefootnote} % poznámky pod tabulkou
\usepackage{color} % barvickyy
\usepackage{array} % schovavani sloupcu v tabulkach

% Pozadovane vstupy pro generovani titulnich stran.
\ThesisAuthor{Barbora Kovalská}
\ThesisSupervisor{Ing. Radoslav Fasuga, Ph.D.}

\CzechThesisTitle{Interaktivní nástroj pro vizualizaci algoritmů a datových struktur}
\EnglishThesisTitle{Interactive Tool for Visualizing Algorithms and Data Structures}

\SubmissionYear{2026}

\ThesisAssignmentFileName{../specification.pdf}

\Acknowledgement{%
Děkuji, že mám ještě tolik času na dokončení této práce.
}

\CzechAbstract{%
Abstraktní čeština.
}
\CzechKeywords{interaktivní nástroj; vizualizace; algoritmy; datové struktury; vzdělávání; simulace}

\EnglishAbstract{%
Abstract English.
}
\EnglishKeywords{interactive tool; visualization; algorithms; data structures; education; simulation}

\AddAcronym{VŠB}{Vysoká škola Báňská}

\addbibresource{resources/sauce.bib}


% Minted things
\let\NewCommand\newcommand
% \immediate\write18{echo $ROAD > .ROAD.tex}
% \immediate\write18{echo $ROAD "nejhorší hack by žožka"}
% \input{.ROAD}
% \immediate\write18{export TEXMF_OUTPUT_DIRECTORY=\ROAD}


\usepackage{minted} %
\setminted{fontsize=\fontsize{9}{11}\selectfont, baselinestretch=1, frame=lines, framesep=8pt, linenos, breaklines}
\renewcommand\listingscaption{Výpis}
\renewcommand\listoflistingscaption{Seznam výpisů zdrojového kódu}

% Uprava hloubky obsahu - pozdeji smazat !
\setcounter{tocdepth}{2}

% Custom reference
\newcommand{\customref}[2]{\hyperref[#2]{#1~\ref*{#2}}}
\newcommand{\chapterref}[1]{(\hyperref[#1]{Kapitola~\ref*{#1}})}
\newcommand{\imageref}[1]{(\hyperref[#1]{Obrázek~\ref*{#1}})}
\newcommand{\tableref}[1]{(\hyperref[#1]{Tabulka~\ref*{#1}})}
\newcommand{\glsref}[1]{\textit{\gls{#1}}}
\newcommand{\gameref}[1]{\textit{#1}}
\newcommand{\attr}[1]{(\texttt{#1})}
\newcommand{\devTool}[2]{\textit{#1}\footnote{\href{#2}{#2}}}

% Hide columns in tables
\newcolumntype{H}{>{\setbox0=\hbox\bgroup}c<{\egroup}@{}}


% Zacatek dokumentu
\begin{document}

% Titulni strany
\MakeTitlePages

% Seznam obrazku
\listoffigures
\clearpage

% Seznam tabulek
\listoftables
\clearpage

% Seznam výpisů zdrojového kódu
\addcontentsline{toc}{chapter}{Seznam výpisů zdrojového kódu}
\listoflistings
\clearpage

% Text
\chapter{Úvod}

Vítejte v mé magisterské práci.


\chapter{Analýza existujících řešení}
\label{chap:other_apps}

V této kapitole jsou popsány vybrané existující aplikace, které slouží k vizualizaci algoritmů a datových struktur. Cílem je analyzovat jejich funkce, uživatelské rozhraní a možnosti interakce, aby bylo možné identifikovat silné a slabé stránky těchto nástrojů. Tato analýza poslouží jako základ pro návrh a implementaci nového interaktivního nástroje, který bude lépe vyhovovat potřebám uživatelů.

Analýza se nejprve zaměří na vizualizaci datových struktur a následně na simulaci třídících algoritmů. Vybrané aplikace budou hodnoceny z hlediska jejich uživatelské přívětivosti, rozsahu podporovaných algoritmů a datových struktur, možnosti interakce a přizpůsobení vizualizací.


\section{Vizualizace datových struktur}
\label{sec:other_apps_data_structures}

Nástroje popsané v této kapitole se zaměřují na vizualizaci různých datových struktur, jako jsou stromy, grafy, zásobníky a fronty.


\subsection{Gnarley Trees}
\label{sec:other_apps_ds_gnarley_trees}

První aplikací, kterou tato práce analyzuje, je \devTool{Gnarley Trees}{https://kubokovac.eu/gnarley-trees}, interaktivní nástroj pro vizualizaci a manipulaci s různými typy stromových datových struktur.

\subsubsection*{Obsah}

Aplikace se zaměřuje především na stromy a haldy, v kontextu různých způsobů ukládání dat a jejich efektivity. U stromů jde například o binární vyhledávací stromy, červeno-černé stromy, AVL stromy, B-stromy a různé další varianty těchto struktur. U front a hald jsou to například prioritní fronty implementované pomocí běžných hald, binárních hald, Fibonacciho hald a dalších.

Každý typ datové struktury má možnost vizualizace základních operací, u stromů například vkládání, mazání a vyhledávání uzlů, u hald a front pak vkládání, inkrementaci hodnoty prvku a odstraňování prvků s nejvyšší prioritou. Dále je zde také možnost vložit předem definovaný počet náhodných prvků a tím rychle zaplnit danou strukturu. Tyto interakce jsou prováděny pomocí uživatelského rozhraní, které umožňuje uživatelům zadávat hodnoty a sledovat, jak se datová struktura mění v reálném čase. Vizualizace je doplněna o animace, které umožňují mimo jiné i krokování operací, což usnadňuje pochopení dynamiky datových struktur. Animace jsou jednoduché, ale efektivní, včetně možnosti krokování vpřed i vzad.

\subsubsection*{Vizuál}

Jedná se o webovou aplikaci napsanou v jazyce Java, vykreslování grafů je realizováno pomocí knihovny CheerpJ. Uživatelské rozhraní je přehledné, umožňuje snadný přístup k různým datovým strukturám a jejich operacím. Kromě toho aplikace ke všemu přikládá vysvětlivky, což je užitečné pro uživatele, kteří se s těmito koncepty teprve seznamují.

Co se týče uživatelského přizpůsobení, stránka má pouze základní stylování, bez možnosti výběru barevných schémat nebo přizpůsobení vzhledu. Stránka také zřejmě má možnost změny jazyka z angličtiny do slovenštiny, ale její zapnutí není na první pohled zřejmé.


\subsection{Data Structure Visualizations}
\label{sec:other_apps_ds_data_structure_visualizations}

Další aplikací je \devTool{Data Structure Visualizations}{https://www.cs.usfca.edu/~galles/visualization/}, výukový nástroj z Univerzity San Francisco, který nabízí vizualizace různých datových struktur a algoritmů.

\subsubsection*{Obsah}

Aplikace pokrývá širokou škálu témat, jak v oblasti datových struktur, tak i algoritmů. Mezi datové struktury patří například zásobníky, fronty, spojové seznamy, stromy (včetně binárních vyhledávacích stromů a hald) a grafy. U každé datové struktury jsou k dispozici vizualizace základních operací, jako je vkládání, mazání a vyhledávání prvků. U algoritmů jsou zahrnuty různé třídící algoritmy (například Bubble Sort, Insertion Sort, Merge Sort) a algoritmy pro prohledávání grafů (například Dijkstrův algoritmus nebo algoritmy na získání kostry grafu).

Interakce s vizualizacemi není sjednocená jako u minulé aplikace, každá datová struktura ji má implementovanou jinak. Animace pro jednotlivé operace jsou velmi jednoduché a oproti předchozí aplikaci postrádají vysvětlivky. Krokování funguje opět vpřed i vzad, včetně možnosti nastavení rychlosti animace. Jako nedostatek lze uvést i absenci možnosti naplnění datové struktury náhodnými hodnotami, což by zrychlilo testování a experimentování s různými scénáři. Třídící algoritmy jsou velmi jednoduché, umožňují uživateli náhodně zamíchat pole čísel, zvolit si třídící algoritmus a sledovat jeho průběh. Jako velmi zajímavé rozšíření aplikace poskytuje možnost napsat si vlastní algoritmus v jazyce JavaScript a vizualizovat jeho průběh.

\subsubsection*{Vizuál}

Aplikace je webová, napsaná v jazyce JavaScript, s jednoduchým uživatelským rozhraním. Celkový vzhled není příliš moderní a uživatelsky přívětivý, ale funkční. Navigace na stránce je mírně komplikovaná, protože uživatel musí procházet různými odkazy, aby se dostal k požadované datové struktuře nebo algoritmu. Ovládání vizualizací je stylováno pomocí tlačítek a vstupních polí, ve stejném duchu jako předchozí aplikace.

Možnosti přizpůsobení jsou omezené, bez možnosti změny barevných schémat nebo vzhledu vizualizací. Responzivita stránky je značně omezená, což může ztížit použití na různých zařízeních. Aplikace je dostupná pouze v angličtině.


\subsection{B Plus Tree}
\label{sec:other_apps_ds_b_plus_tree}

Poslední aplikací zaměřující se na datové struktury, jejíž analýzu tato práce provede, je \devTool{B Plus Tree}{https://bplustree.app/}, interaktivní nástroj pro vizualizaci B+ stromů.

\subsubsection*{Obsah}

Aplikace se specializuje výhradně na B+ stromy, neobsahuje tedy jiné datové struktury. U B+ stromů jsou k dispozici vizualizace základních operací, jako je vkládání, mazání a vyhledávání klíčů. Mimo tyto poskytuje i možnost nastavit danému stromu maximální počet klíčů na vnitřní uzel nebo list a také možnost generování náhodných klíčů pro rychlé naplnění stromu. Chybí zde vysvětlivky k jednotlivým operacím, krátký popis struktury se věnuje spíše využití B+ stromů v databázových systémech.

Interakce s vizualizacemi je intuitivní, uživatel může zadávat hodnoty prostřednictvím vstupních polí a sledovat, jak se strom mění v reálném čase. Animace nemají možnost krokování, po provedení operace se strom aktualizuje rovnou do nového stavu, a také je není možné krokovat zpět.

\subsubsection*{Vizuál}

Tato aplikace je rovněž webová, s velmi jednoduchým ale moderním uživatelským rozhraním. Obsah je tak malý, že není potřeba složitější navigace, vše je dostupné přímo na hlavní stránce. Ovládání vizualizací je realizováno pomocí tlačítek a vstupních polí. Stránka nenabízí žádné možnosti přizpůsobení vzhledu nebo změnu jazyka. Aplikace je velmi responzivní a funguje dobře na různých zařízeních.


\subsection{Porovnání}
\label{sec:other_apps_ds_comparison}

V této sekci jsou shrnuty klady a zápory jednotlivých analyzovaných aplikací zaměřených na vizualizaci datových struktur. Následující tabulka \tableref{tab:other_apps_ds_comparison} poskytuje přehledné srovnání jejich vlastností.

\begin{table}[H]
    \centering
    \resizebox{\textwidth}{!}{%
    \begin{tabular}{l c c c}
        \toprule
        \textbf{Vlastnost} & \textbf{Gnarley Trees} & \textbf{Data Structure Visualizations} & \textbf{B Plus Tree} \\
        \midrule
        Podpora více datových struktur & Ano & Ano & Ne \\
        Uživatelské přizpůsobení & Omezené & Omezené & Žádné \\
        Krokování animací & Ano (vpřed i vzad) & Ano (vpřed i vzad) & Ne \\
        Vysvětlivky k operacím & Ano & Ne & Ne \\
        Generování náhodných dat & Ano & Ne & Ano \\
        Responzivita & Dobrá & Slabá & Výborná \\
        Jazyková podpora & Angličtina, slovenština & Angličtina & Angličtina \\
        \bottomrule
    \end{tabular}}
    \caption{Porovnání analyzovaných aplikací pro vizualizaci datových struktur}
    \label{tab:other_apps_ds_comparison}
\end{table}



\section{Simulace třídících algoritmů}
\label{sec:other_apps_sorting_algorithms}

Tato kapitola se věnuje aplikacím, které umožňují vizualizaci a simulaci různých třídících algoritmů, jako jsou Bubble Sort, Quick Sort, Merge Sort a další.

\subsection{Sort Visualizer}
\label{sec:other_apps_sa_sort_visualizer}

\subsection{VisuAlgo}
\label{sec:other_apps_sa_visualgo}

\subsection{Toptal Sorting Algorithms Animations}
\label{sec:other_apps_sa_toptal}

\subsection{Porovnání}
\label{sec:other_apps_sa_comparison}

\section{Shrnutí}
\label{sec:other_apps_summary}


\chapter{Závěr}

Tady končíme, všichni ven.


% Seznam literatury
\printbibliography[title={Literatura}, heading=bibintoc]

% Přílohy
\appendix

\end{document}
