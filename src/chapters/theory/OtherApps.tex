\chapter{Analýza existujících řešení}
\label{chap:other_apps}

V této kapitole jsou popsány vybrané existující aplikace, které slouží k vizualizaci algoritmů a datových struktur. Cílem je analyzovat jejich funkce, uživatelské rozhraní a možnosti interakce, aby bylo možné identifikovat silné a slabé stránky těchto nástrojů. Tato analýza poslouží jako základ pro návrh a implementaci nového interaktivního nástroje, který bude lépe vyhovovat potřebám uživatelů.

Analýza se nejprve zaměří na vizualizaci datových struktur a následně na simulaci třídících algoritmů. Vybrané aplikace budou hodnoceny z hlediska jejich uživatelské přívětivosti, rozsahu podporovaných algoritmů a datových struktur, možnosti interakce a přizpůsobení vizualizací.


\section{Vizualizace datových struktur}
\label{sec:other_apps_data_structures}

Nástroje popsané v této kapitole se zaměřují na vizualizaci různých datových struktur, jako jsou stromy, grafy, zásobníky a fronty.


\subsection{Gnarley Trees}
\label{sec:other_apps_ds_gnarley_trees}

První aplikací, kterou tato práce analyzuje, je \devTool{Gnarley Trees}{https://kubokovac.eu/gnarley-trees}, interaktivní nástroj pro vizualizaci a manipulaci s různými typy stromových datových struktur.

\subsubsection*{Obsah}

Aplikace se zaměřuje především na stromy a haldy, v kontextu různých způsobů ukládání dat a jejich efektivity. U stromů jde například o binární vyhledávací stromy, červeno-černé stromy, AVL stromy, B-stromy a různé další varianty těchto struktur. U front a hald jsou to například prioritní fronty implementované pomocí běžných hald, binárních hald, Fibonacciho hald a dalších.

Každý typ datové struktury má možnost vizualizace základních operací, u stromů například vkládání, mazání a vyhledávání uzlů, u hald a front pak vkládání, inkrementaci hodnoty prvku a odstraňování prvků s nejvyšší prioritou. Dále je zde také možnost vložit předem definovaný počet náhodných prvků a tím rychle zaplnit danou strukturu. Tyto interakce jsou prováděny pomocí uživatelského rozhraní, které umožňuje uživatelům zadávat hodnoty a sledovat, jak se datová struktura mění v reálném čase. Vizualizace je doplněna o animace, které umožňují mimo jiné i krokování operací, což usnadňuje pochopení dynamiky datových struktur. Animace jsou jednoduché, ale efektivní, včetně možnosti krokování vpřed i vzad.

\subsubsection*{Vizuál}

Jedná se o webovou aplikaci napsanou v jazyce Java, vykreslování grafů je realizováno pomocí knihovny CheerpJ. Uživatelské rozhraní je přehledné, umožňuje snadný přístup k různým datovým strukturám a jejich operacím. Kromě toho aplikace ke všemu přikládá vysvětlivky, což je užitečné pro uživatele, kteří se s těmito koncepty teprve seznamují.

Co se týče uživatelského přizpůsobení, stránka má pouze základní stylování, bez možnosti výběru barevných schémat nebo přizpůsobení vzhledu. Stránka také zřejmě má možnost změny jazyka z angličtiny do slovenštiny, ale její zapnutí není na první pohled zřejmé.


\subsection{Data Structure Visualizations}
\label{sec:other_apps_ds_data_structure_visualizations}

Další aplikací je \devTool{Data Structure Visualizations}{https://www.cs.usfca.edu/~galles/visualization/}, výukový nástroj z Univerzity San Francisco, který nabízí vizualizace různých datových struktur a algoritmů.

\subsubsection*{Obsah}

Aplikace pokrývá širokou škálu témat, jak v oblasti datových struktur, tak i algoritmů. Mezi datové struktury patří například zásobníky, fronty, spojové seznamy, stromy (včetně binárních vyhledávacích stromů a hald) a grafy. U každé datové struktury jsou k dispozici vizualizace základních operací, jako je vkládání, mazání a vyhledávání prvků. U algoritmů jsou zahrnuty různé třídící algoritmy (například Bubble Sort, Insertion Sort, Merge Sort) a algoritmy pro prohledávání grafů (například Dijkstrův algoritmus nebo algoritmy na získání kostry grafu).

Interakce s vizualizacemi není sjednocená jako u minulé aplikace, každá datová struktura ji má implementovanou jinak. Animace pro jednotlivé operace jsou velmi jednoduché a oproti předchozí aplikaci postrádají vysvětlivky. Krokování funguje opět vpřed i vzad, včetně možnosti nastavení rychlosti animace. Jako nedostatek lze uvést i absenci možnosti naplnění datové struktury náhodnými hodnotami, což by zrychlilo testování a experimentování s různými scénáři. Třídící algoritmy jsou velmi jednoduché, umožňují uživateli náhodně zamíchat pole čísel, zvolit si třídící algoritmus a sledovat jeho průběh. Jako velmi zajímavé rozšíření aplikace poskytuje možnost napsat si vlastní algoritmus v jazyce JavaScript a vizualizovat jeho průběh.

\subsubsection*{Vizuál}

Aplikace je webová, napsaná v jazyce JavaScript, s jednoduchým uživatelským rozhraním. Celkový vzhled není příliš moderní a uživatelsky přívětivý, ale funkční. Navigace na stránce je mírně komplikovaná, protože uživatel musí procházet různými odkazy, aby se dostal k požadované datové struktuře nebo algoritmu. Ovládání vizualizací je stylováno pomocí tlačítek a vstupních polí, ve stejném duchu jako předchozí aplikace.

Možnosti přizpůsobení jsou omezené, bez možnosti změny barevných schémat nebo vzhledu vizualizací. Responzivita stránky je značně omezená, což může ztížit použití na různých zařízeních. Aplikace je dostupná pouze v angličtině.


\subsection{B Plus Tree}
\label{sec:other_apps_ds_b_plus_tree}

Poslední aplikací zaměřující se na datové struktury, jejíž analýzu tato práce provede, je \devTool{B Plus Tree}{https://bplustree.app/}, interaktivní nástroj pro vizualizaci B+ stromů.

\subsubsection*{Obsah}

Aplikace se specializuje výhradně na B+ stromy, neobsahuje tedy jiné datové struktury. U B+ stromů jsou k dispozici vizualizace základních operací, jako je vkládání, mazání a vyhledávání klíčů. Mimo tyto poskytuje i možnost nastavit danému stromu maximální počet klíčů na vnitřní uzel nebo list a také možnost generování náhodných klíčů pro rychlé naplnění stromu. Chybí zde vysvětlivky k jednotlivým operacím, krátký popis struktury se věnuje spíše využití B+ stromů v databázových systémech.

Interakce s vizualizacemi je intuitivní, uživatel může zadávat hodnoty prostřednictvím vstupních polí a sledovat, jak se strom mění v reálném čase. Animace nemají možnost krokování, po provedení operace se strom aktualizuje rovnou do nového stavu, a také je není možné krokovat zpět.

\subsubsection*{Vizuál}

Tato aplikace je rovněž webová, s velmi jednoduchým ale moderním uživatelským rozhraním. Obsah je tak malý, že není potřeba složitější navigace, vše je dostupné přímo na hlavní stránce. Ovládání vizualizací je realizováno pomocí tlačítek a vstupních polí. Stránka nenabízí žádné možnosti přizpůsobení vzhledu nebo změnu jazyka. Aplikace je velmi responzivní a funguje dobře na různých zařízeních.


\subsection{Porovnání}
\label{sec:other_apps_ds_comparison}

V této sekci jsou shrnuty klady a zápory jednotlivých analyzovaných aplikací zaměřených na vizualizaci datových struktur. Následující tabulka \tableref{tab:other_apps_ds_comparison} poskytuje přehledné srovnání jejich vlastností.

\begin{table}[H]
    \centering
    \resizebox{\textwidth}{!}{%
    \begin{tabular}{l c c c}
        \toprule
        \textbf{Vlastnost} & \textbf{Gnarley Trees} & \textbf{Data Structure Visualizations} & \textbf{B Plus Tree} \\
        \midrule
        Podpora více datových struktur & Ano & Ano & Ne \\
        Uživatelské přizpůsobení & Omezené & Omezené & Žádné \\
        Krokování animací & Ano (vpřed i vzad) & Ano (vpřed i vzad) & Ne \\
        Vysvětlivky k operacím & Ano & Ne & Ne \\
        Generování náhodných dat & Ano & Ne & Ano \\
        Responzivita & Dobrá & Slabá & Výborná \\
        Jazyková podpora & Angličtina, slovenština & Angličtina & Angličtina \\
        \bottomrule
    \end{tabular}}
    \caption{Porovnání analyzovaných aplikací pro vizualizaci datových struktur}
    \label{tab:other_apps_ds_comparison}
\end{table}

Dále byla provedena analýza a porovnání konkrétních datových struktur, které výše zmíněné aplikace podporují. \customref{Tabulka}{tab:other_apps_ds_structures} shrnuje, které datové struktury jsou v jednotlivých aplikacích podporovány. 

\begin{table}[H]
    \centering
    \resizebox{\textwidth}{!}{%
    \begin{tabular}{l c c c}
        \toprule
        \textbf{Datová struktura} & \textbf{Gnarley Trees} & \textbf{Data Structure Visualizations} & \textbf{B Plus Tree} \\
        \midrule
        Binární vyhledávací strom   & Ano & Ano & Ne \\
        AVL strom                   & Ano & Ano  & Ne \\
        Červeno-černý strom         & Ano & Ano  & Ne \\
        B-strom                     & Ano & Ano  & Ne \\
        B+ strom                    & Ne  & Ano  & Ano \\
        Spojový seznam              & Ano\tablefootnote{Pouze v teoretické části.\label{t}} & Ne & Ne \\
        Zásobník                    & Ano\footref{t} & Ano & Ne \\
        Fronta                      & Ano\footref{t}  & Ano & Ne \\
        Halda                       & Ano  & Ano & Ne \\
        \bottomrule
    \end{tabular}}
    \caption{Podporované datové struktury v analyzovaných aplikacích}
    \label{tab:other_apps_ds_structures}
\end{table}

Z uvedeného shrnutí je patrné, že nejkomplexnější podporu datových struktur nabízí aplikace \textit{Data Structure Visualizations}, která pokrývá téměř všechny uvedené struktury. \textit{Gnarley Trees} nabízí také širokou škálu datových struktur, ale některé z nich jsou dostupné pouze v teoretické části aplikace a nejsou vizualizovány. Nejméně obsáhlá je aplikace \textit{B Plus Tree}, která se sice specializuje pouze na B+ stromy, ale zato je v této oblasti velmi dobře propracovaná.


\section{Simulace třídících algoritmů}
\label{sec:other_apps_sorting_algorithms}

Tato kapitola se věnuje aplikacím, které umožňují vizualizaci a simulaci různých třídících algoritmů, jako jsou Bubble Sort, Quick Sort, Merge Sort a další.

\subsection{Sort Visualizer}
\label{sec:other_apps_sa_sort_visualizer}

Aplikace \devTool{Sort Visualizer}{https://www.sortvisualizer.com/} je interaktivní nástroj pro vizualizaci různých třídících algoritmů, který umožňuje uživatelům sledovat průběh třídění v reálném čase spolu s popisem algoritmů a jejich časovou a prostorovou složitostí.

\subsubsection*{Obsah}

Sort Visualizer nabízí simulace několika běžných třídících algoritmů, které dělí na logaritmické (Quick Sort, Merge Sort, Heap Sort), kvadratické (Bubble Sort, Insertion Sort, Selection Sort) a další (např. Radix Sort). Přidává také možnost, jak si uživatel může vytvořit vlastní algoritmus pomocí jazyka JavaScript a vizualizovat jeho průběh.

Interakce se simulací je značně omezená, uživatel si může pouze zvolit algoritmus, velikost pole a rychlost animace. Animace spočívá v zobrazení sloupců reprezentujících hodnoty v poli, které se mění podle průběhu třídění. Zajímavé zpestření animace jsou zvukové efekty, které doprovázejí porovnávání a výměny prvků a svým tónem reprezentují hodnotu daného prvku. Krokování animace není možné.

Každý algoritmus je doplněn o krátký popis jeho principu, časové složitosti v nejlepším, průměrném a nejhorším případě a také o prostorovou složitost. Přiloženy jsou také zdrojové kódy daného algoritmu v několika programovacích jazycích. Zdrojové kódy jsou však statické, nelze je během simulace nijak měnit a také nejsou zvýrazněny části kódu, které se právě vykonávají.

\subsubsection*{Vizuál}

Tato webová aplikace je napsaná pomocí webového frameworku Flask, s moderním a přehledným uživatelským rozhraním. Navigace na stránce je jednoduchá, všechny třídící algoritmy jsou vždy dostupné z navigačního panelu. Samotná vizualizace je vizuálně velmi poutavá a v doprovodu se zvukovými efekty působí přívětivě.

Aplikace nenabízí žádné možnosti přizpůsobení vzhledu nebo změnu jazyka. Výchozí barevné schéma je v temném režimu. Responzivita stránky je skvělá, stránka funguje dobře téměř v jakémkoliv rozlišení.


\subsection{VisuAlgo}
\label{sec:other_apps_sa_visualgo}

Jedná se o další webovou aplikaci, sloužící jako výukový nástroj pro vizualizaci všemožných algoritmů a datových struktur. Aplikace \devTool{VisuAlgo}{https://visualgo.net/en} nabízí širokou škálu vizualizací, včetně třídících algoritmů.

\subsubsection*{Obsah}

Rozsah této aplikace je velmi široký, pokrývá mnoho témat v oblasti algoritmů a datových struktur. Stránka nabízí vizualizace pro různé datové struktury, jako jsou zásobníky, fronty, spojové seznamy, stromy a grafy. Kromě toho obsahuje i vizualizace různých algoritmů, včetně třídících algoritmů, algoritmů pro prohledávání grafů a dalších. Obsahuje i interaktivní cvičení a kvízy, které pomáhají uživatelům lépe pochopit koncepty.

V kontextu třídících algoritmů nabízí vizualizace pro algoritmy jako Bubble Sort, Insertion Sort, Selection Sort, Merge Sort, Quick Sort a Heap Sort. Animace jsou jednoduše znázorněny opět pomocí sloupců reprezentujících hodnoty v poli. Uživatel má možnosti nastavení rychlosti přehrávání, plné krokování animace vpřed i vzad a také možnost pozastavení animace. Kromě toho aplikace poskytuje i podrobné vysvětlení principu každého algoritmu, včetně jeho časové a prostorové složitosti. Během animace je přístupný i pseudokód algoritmu, který je zvýrazňován podle aktuálně vykonávané části.

\subsubsection*{Vizuál}

Webové stránky mají značně zaplněné uživatelské rozhraní, což může být pro nové uživatele trochu matoucí. Navigace na stránce je relativně dobře organizovaná, s jasnými odkazy na různé sekce a témata. Stránka s vizualizacemi se může zdát přeplněná ještě více, protože obsahuje mnoho ovládacích prvků a informací. Toto okno také obsahuje vysvětlivky, které jsou provedeny ve vyskakovacích oknech, které uživatele zahltí při prvním použití.

Výchozí barevné schéma je světlé, uživatel si nemůže změnit režim na tmavý. Responzivita stránky je dobrá, když se rozlišení zmenší na nedostatečnou šířku, aplikace uživatele upozorní, že vizualizace nemusí fungovat správně. Aplikace podporuje změnu jazyka, na výběr uživatelé mají z angličtiny, čínštiny a indonéštiny. 


\subsection{Toptal Sorting Algorithms Animations}
\label{sec:other_apps_sa_toptal}

Poslední analyzovanou aplikací je \devTool{Toptal Sorting Algorithms Animations}{https://www.toptal.com/developers/sorting-algorithms}, téměř jednostránkový webový nástroj pro vizualizaci a porovnání různých třídících algoritmů.

\subsubsection*{Obsah}

Aplikace nabízí vizualizace pro několik běžných třídících algoritmů, které již byly zmíněny výše. Jednotlivé algoritmy jsou umístěny v matici, kde každý sloupec reprezentuje jeden algoritmus a každý řádek představuje typ vstupních dat (např. náhodná data, téměř seřazená data nebo obráceně seřazená data). Pomocí tohoto rozložení může uživatel snadno porovnat výkon různých algoritmů na různých typech vstupů.

Interakce s vizualizacemi je velmi omezená, uživatel si může pouze zvolit algoritmus a typ vstupních dat. Animace jsou jednoduché, zobrazují sloupce reprezentující hodnoty v poli, které se mění podle průběhu třídění. Krokování animace není možné.

Uživatel má také možnost zobrazit si detail konkrétního algoritmu, kde je k němu přiložena vysvětlivka, pseudokód a časová a prostorová složitost. Rozkliknutí konkrétního datového typu zase zobrazí konkrétní porovnání časové složitosti všech algoritmů na daném typu vstupu a výčet poznatků o jejich výkonu.

\subsubsection*{Vizuál}

Jedná se o velmi jednoduchou webovou aplikaci. Hlavní stránka obsahuje pouze výše uvedenou matici algoritmů a typů vstupních dat spolu s obecnou diskuzí na téma třídících algoritmů. Detail algoritmu je veden ve stejném stylu, animace jsou pouze větší a jsou doplněny o pseudokód a diskuzi ke konkrétnímu algoritmu.

Stránka nemá žádné možnosti přizpůsobení vzhledu nebo změnu jazyka. Výchozí barevné schéma je světlé. Jako jediná z dosud analyzovaných aplikací také obsahuje reklamy. Responzivita stránky je dobrá, funguje dobře na různých zařízeních. 


\subsection{Porovnání}
\label{sec:other_apps_sa_comparison}

V této sekci jsou shrnuty klady a zápory jednotlivých analyzovaných aplikací zaměřených na vizualizaci třídících algoritmů. Následující tabulka \tableref{tab:other_apps_sa_comparison} poskytuje přehledné srovnání jejich vlastností.

\begin{table}[H]
    \centering
    \resizebox{\textwidth}{!}{%
    \begin{tabular}{l c c c c}
        \toprule
        \textbf{Vlastnost} & \textbf{Sort Visualizer} & \textbf{VisuAlgo} & \textbf{Toptal Sorting Algorithms Animations} \\
        \midrule
        Uživatelské přizpůsobení & Žádné & Žádné & Žádné \\
        Krokování animací & Ne & Ano (vpřed i vzad) & Ne \\
        Vysvětlivky k operacím & Ano & Ano & Ano \\
        Zvukové efekty & Ano & Ne & Ne \\
        Responzivita & Výborná & Dobrá & Dobrá \\
        Jazyková podpora & Angličtina & Angličtina, čínština, indonéština & Angličtina \\
        \bottomrule
    \end{tabular}}
    \caption{Porovnání analyzovaných aplikací pro vizualizaci třídících algoritmů}
    \label{tab:other_apps_sa_comparison}
\end{table}

Z konkrétní analýzy vyšlo najevo, že všechny tři aplikace podporují základní množinu nejznámějších třídících algoritmů, ale liší se v možnostech interakce a uživatelského přizpůsobení. \textit{VisuAlgo} nabízí nejvíce možností interakce, včetně krokování animací a podrobného pseudokódu, zatímco \textit{Sort Visualizer} a \textit{Toptal Sorting Algorithms Animations} mají omezené možnosti interakce. Zvukové efekty jsou unikátní funkcí \textit{Sort Visualizer}, která přidává další vrstvu zážitku z vizualizace.


\section{Shrnutí}
\label{sec:other_apps_summary}

Analýza existujících řešení, provedená v této kapitole, poskytla cenné poznatky o současných nástrojích pro vizualizaci datových struktur a třídících algoritmů. Z analýzy vyplynulo, že zatímco některé aplikace nabízejí širokou škálu funkcí a interakcí, jiné se zaměřují na specifické datové struktury nebo algoritmy s omezenými možnostmi přizpůsobení.

Na základě porovnání podporovaných datových struktur bylo rozhodnuto, že modelová aplikace bude zahrnovat vizualizaci následujících datových struktur: binární vyhledávací strom, AVL strom, červeno-černý strom, B-strom, B+ strom, spojový seznam, zásobník, fronta a halda. Tato volba byla učiněna s ohledem na jejich význam v oblasti informatiky a jejich časté využití v praxi.

Co se týče třídících algoritmů, modelová aplikace bude podporovat vizualizaci následujících algoritmů: Bubble Sort, Insertion Sort, Selection Sort, Merge Sort, Quick Sort a Heap Sort. Tyto algoritmy byly vybrány na základě jejich rozšířenosti a různorodosti přístupů k třídění dat.

Zdůrazněna byla také potřeba uživatelsky přívětivého rozhraní, které umožní snadnou navigaci a interakci s vizualizacemi. Dále bylo identifikováno několik klíčových funkcí, které by měly být zahrnuty do modelové aplikace, včetně možnosti krokování animací, generování náhodných dat a poskytování vysvětlivek k jednotlivým operacím.




% modelová aplikace?