\chapter{Analýza existujících řešení}
\label{chap:other_apps}

V této kapitole jsou popsány vybrané existující aplikace, které slouží k vizualizaci algoritmů a datových struktur. Cílem je analyzovat jejich funkce, uživatelské rozhraní a možnosti interakce, aby bylo možné identifikovat silné a slabé stránky těchto nástrojů. Tato analýza poslouží jako základ pro návrh a implementaci nového interaktivního nástroje, který bude lépe vyhovovat potřebám uživatelů.

Analýza se nejprve zaměří na vizualizaci datových struktur a následně na simulaci třídících algoritmů. Vybrané aplikace budou hodnoceny z hlediska jejich uživatelské přívětivosti, rozsahu podporovaných algoritmů a datových struktur, možnosti interakce a přizpůsobení vizualizací.


\section{Vizualizace datových struktur}
\label{sec:other_apps_data_structures}

Nástroje popsané v této kapitole se zaměřují na vizualizaci různých datových struktur, jako jsou stromy, grafy, zásobníky a fronty.

\subsection{Gnarley Trees}
\label{sec:other_apps_ds_gnarley_trees}

První aplikací, kterou tato práce analyzuje, je \devTool{Gnarley Trees}{https://kubokovac.eu/gnarley-trees}, interaktivní nástroj pro vizualizaci a manipulaci s různými typy stromových datových struktur.

\subsubsection*{Obsah}

Aplikace se zaměřuje především na stromy a haldy, v kontextu různých způsobů ukládání dat a jejich efektivity. U stromů jde například o binární vyhledávací stromy, červeno-černé stromy, AVL stromy, B-stromy a různé další varianty těchto struktur. U front a hald jsou to například prioritní fronty implementované pomocí běžných hald, binárních hald, Fibonacciho hald a dalších.

Každý typ datové struktury má možnost vizualizace základních operací, u stromů například vkládání, mazání a vyhledávání uzlů, u hald a front pak vkládání, inkrementaci hodnoty prvku a odstraňování prvků s nejvyšší prioritou. Tyto interakce jsou prováděny pomocí uživatelského rozhraní, které umožňuje uživatelům zadávat hodnoty a sledovat, jak se datová struktura mění v reálném čase. Vizualizace je doplněna o animace, které umožňují mimo jiné i krokování operací, což usnadňuje pochopení dynamiky datových struktur. Animace jsou jednoduché, ale efektivní, včetně možnosti krokování vpřed i vzad.

\subsection{Data Structure Visualizations}
\label{sec:other_apps_ds_data_structure_visualizations}

\subsection{B Plus Tree}
\label{sec:other_apps_ds_b_plus_tree}

\subsection{Porovnání}
\label{sec:other_apps_ds_comparison}


\section{Simulace třídících algoritmů}
\label{sec:other_apps_sorting_algorithms}

Tato kapitola se věnuje aplikacím, které umožňují vizualizaci a simulaci různých třídících algoritmů, jako jsou Bubble Sort, Quick Sort, Merge Sort a další.

\subsection{Sort Visualizer}
\label{sec:other_apps_sa_sort_visualizer}

\subsection{VisuAlgo}
\label{sec:other_apps_sa_visualgo}

\subsection{Toptal Sorting Algorithms Animations}
\label{sec:other_apps_sa_toptal}

\subsection{Porovnání}
\label{sec:other_apps_sa_comparison}

\section{Shrnutí}
\label{sec:other_apps_summary}
